\documentclass[10pt,twoside,twocolumn]{book}
\usepackage[bg-letter]{lib/rpg-book} % Options: bg-a4, bg-letter, bg-full, bg-print, bg-none.
\usepackage[english]{babel}
\usepackage[utf8]{inputenc}
\usepackage[hidelinks]{hyperref}

\usepackage[T1]{fontenc}
\usepackage{tgcursor}
\usepackage{xcolor}

\usepackage{rotating}
\usepackage{tikz}

\makeatletter
\newcommand{\globalcolor}[1]{%
  \color{#1}\global\let\default@color\current@color
}
\makeatother

\makeatletter
\def\newsect{%
    \vskip2em plus0.5em minus0.5em%
    \hbox to\linewidth{\hfil*\quad\quad*\quad\quad*\hfil}%
    \penalty10000\vskip2em plus0.5em minus0.5em%
    \@afterindentfalse\@afterheading%
}
\makeatother%


\definecolor{alien}{rgb}{1, 1, 0.9}

\definecolor{PC}{HTML}{c0ccc1}


\newcommand\pc[1]{\colorbox{PC}{\color{black} #1}}

%  

\AtBeginDocument{\globalcolor{alien}}

\title{Melting Point}
\date{\today}
\author{Arthur Marques \\ u/marques\_art\_boris}





% Start document
\begin{document}
\fontfamily{ppl}\selectfont % Set text font
\frontmatter

\maketitle



\begin{rpg-warnbox}{A note form the author}
  This scenario discusses terrorism, civil rights, and some topics sensitive to some people. If you are not comfortable with that, this game might not be for you.

  \begin{flushright}
  -- Arthur Marques
  \end{flushright}
\end{rpg-warnbox}


\tableofcontents

% Your content goes here
\mainmatter
% \chapter{Chapter name}

% \section{Section name}
% \lipsum[1] % filler text

% \subsection{Subsection name}
% \subsubsection{subsubsection name}

% \begin{rpg-commentbox}{rpg-commentbox name}
% 	you can add some comments using this box
% \end{rpg-commentbox}

% \begin{rpg-warnbox}{rpg-warnbox name}
% 	you can add some warnings using this box
% \end{rpg-warnbox}

% \begin{rpg-quotebox}{rpg-quotebox name}
%     this is a quote box
% \end{rpg-quotebox}

% %\newpage % Acts as columbreak because of twocolumn option; for pagebreak use \clearpage

% \header{default rpg-table (2 column)}
% \begin{rpg-table}
%    	\textbf{Table head 1}  & \textbf{Table head 2} \\
%    	Some value  & Some value \\
%    	Some value  & Some value \\
%    	Some value  & Some value
% \end{rpg-table}

% % For more columns, you can say \begin{rpg-table}[your options here].
% % For instance, if you wanted three columns, you could say
% % \begin{rpg-table}[XXX]. The usual host of tabular parameters are
% % aailable as well.
% \header{rpg-table with more columns}
% \begin{rpg-table}[XXX]
%     \textbf{Table head 1}  & \textbf{Table head 2} & \textbf{Table head 3}\\
%    	Some value  & Some value & Some value\\
%    	Some value  & Some value & Some value\\
%    	Some value  & Some value & Some value
% \end{rpg-table}

% \newsect

\chapter{Situation}



\section{Scenario Overview}



\begin{rpg-commentbox}{Overview}
    ``\textit{Be a real shame if the lowest bidder on that fire safety system cut corners to increase their profit margin.}''

    \begin{flushright}
    -- Joker4-1
    \end{flushright}

    A space shuttle ambulance brings a high-priority patient to San Cristobal Medical Facility. The shuttle belongs to Weyland-Yutani crew and they had pulled strings to avoid any security clearances or quarantine measures expected by normal crew. Whoever or whatever they bring is high enough in the food chain that the facility makes room for them. It is not like shift workers who suffered any accidents need a hospital bed.
    
    The patient has a xenomorph embryo growing inside them and  Weyland-Yutani is eager to extract the asset. This would have been a profitable day for the corporation if not by a second xenomorph who has lurked in the shuttle waiting for an easy prey. As the shuttle lifts, the alien strikes embracing the pilot---in the now pressurized cabin---to their death. Adrift, the shuttle violently clashes in the landing area causing a
    huge blast that propagates through the facility.
\end{rpg-commentbox}


\begin{rpg-commentbox}{Fire Brigade}
    Players are firefighters that now need to assess the situation and suppress 
    the raging fire before the station's computer---APOLLO---decides that the area is too hazardous and that it should be detached from the main structure before it compromises the entire station.

    There are many things at stake, such as the lives of skilled crew, 
    scientific data from assorted companies and billions of dollars in company assets. As a firefighter, this is the worst imaginable situation, and some of 
    you know that sacrifices will be made.
\end{rpg-commentbox}


\newsect

\begin{rpg-commentbox}{Before the firefighters arrival}
    Living in space is hell and everyone knows that. Most of the staff that could have escaped tha facility has already left, carrying low-risk patients that could have held their own with them. 

    Unfortunately, security measures from the facility also took place sealing doors in hope that the fire suppression system can minimize the fire/damage. 
    This has made several sections of the facility inaccessible and only skilled firefighters can make their way throughout the chaos.
\end{rpg-commentbox}


\newsect


\begin{figure}[!b]
    \centering
    \includegraphics[width=.45\textwidth]{img/bg/crew.jpg}
\end{figure}


\clearpage



\medskip
\begin{rpg-commentbox}{Firefighting in Space}
\begin{small}
    \textit{Firefighters usually only enter a building when it is safe enough to do so. Their priority is to search for survivors, find the source of the fire so they can assess how to better suppress it as well as find any power or electrical sources that must be disabled so that they can act as best as they can.}

    \textit{Visibility is one of the major problems while sweeping a building. For the purposes of the game, it is an impeding factor, but one that is not as accurate as it should be. Due to smoke, firefighters usually crouch and stay as close to the floor as they can. This is also to sweep the floor in front of them, find any victims, and avoid stumbling over any dangers.}

    \textit{There is a lot of cordination among firefighters and a ``chief'' assess everything from a safe place where he can call the shots and request firefighters to evacuate before they are trapped in the fire themselves, give priority to a more important matter, etc.}

    \textit{A potential complication on using water to suppress fires in a space station is that it can cause more electrical fires, or damage corporate equipment. For small electrical panels, fire extinguishers are a common tool.For bigger areas engulfed by the fire, standard procedures are to seal blocks of a building and vent all the air away, or to pray that the gaseous fire suppression systems have done their job. }

    \textit{Accidentally sucking all the air from a building is also not something that should be easily accessible---you don't want a toddler pulling a lever and killing dozens of people. Hence, these security measures are placed under well sealed hatches that can be pried open by a ``haligan'' tool or a welding torch.  }

    \textit{Sealing all doors before pulling the vent mechanism in large rooms is extremely problematic and this might force space firefighters to split up so that they can cover such large rooms. For example, in a T shape corridor, one firefighter stays near the entrance while the other rushes, closes the middle door and makes his way to the other side. As time is valuable, the firefighter proceeds to the next room, closes the door behind them and as soon as that is done, the firefighter who stayed behind pulls the lever venting all air out of that section. Once it is safe to proceed, they will meet the firefighter in the next room who is already taking similar measures.}

    \textit{What about victims? Crew might be expandable and corporate assets are often a priority. Standard procedure is to assess a victim's patch and check their role. Scientists or high corporate staff are too valuable and whenever possible, should be rescued. Unfortunately, the same is not true for roughnecks or security personnel. This puts firefighters in an odd position in a space station, some have made personal connections or even relationships that often clash with their role. Notably, roughnecks appreciate the job of  firefighters when it's just an everyday heavy machinery accident but they loathe them in the aftermath of a fire. Although rotating through stations so that any griefs between crew and firefighters can heal is not uncommon, sometimes this is not possible and firefighters face several mental health problems.}

\end{small}
\end{rpg-commentbox}


\newsect

\medskip


\medskip
\begin{rpg-commentbox}{Fire Supression}
\begin{small}
    \textit{Gaseous fire suppression, also called clean agent fire suppression, is a term to describe the use of inert gases and chemical agents to extinguish a fire.}    

    \textit{Systems working on a total flooding principle apply an extinguishing agent to a three dimensional enclosed space in order to achieve a concentration of the agent (volume percent of the agent in air) adequate to extinguish the fire. These types of systems may be operated automatically by detection and related controls or manually by the operation of a system actuator.}

\end{small}
\end{rpg-commentbox}

\newsect


\medskip
\begin{rpg-commentbox}{Halon}
\begin{small}
    \textit{Halon is a liquefied, compressed gas that stops the spread of fire by chemically disrupting combustion.}

    \textit{Halon leaves no residue and doesn't generally destroy sensitive items like paper and electronics, unlike foam, water, CO2, dry chem, etc. It's a nice perk to be able to suppress fire in a data center without having to write off ten million dollars worth of server equipment because it was destroyed by whatever you used to put out the fire.}

    \textit{While the two currently used types of halon gas are not generally considered deadly, they can still produce toxic by-products as they work to extinguish a fire.}

    \textit{High concentrations of halon can create an oxygen-deficient environment. This can cause people to suffocate.}
\end{small}
\end{rpg-commentbox}


\clearpage


\begin{rpg-commentbox}{Mainframes and Synthetics}
\begin{small}
    \textit{Synthetics do not breath and do not suffer from stress. One would expect that they would be used for such hazardous endeavor. 
    However, they are \textbf{expensive} and not easily replaceable, so only defective synthetics are ever provided to a fire station. }


    \textit{The rudimentary working Joes 
    sold by Seegson Corporation are also not a viable alternative. Their motor system is lack lusting making them move at a slow pace, and their neural synapses do not respond well to complex situations. Usually working Joes just drag victims out of a building whenever a fire alarm rings. Seegson Corporation answers to several law suits from people trying to get disability benefits after a working Joe dragged them to safety.}

    \textit{When a fire does not damage the structural integrity of a building, MU/TH/UR or APOLLO often perform venting supression procedures automatically. In case of damages, the system provides security footage that a fire marshall can use for tactical assessment. Nonetheless, security footage is often not available because this can be used for liability. Insurance companies often seek footage and sometimes even install rogue monitoring systems so they do not pay exorbitant values in the ever moving corporate chess.}

\end{small}
\end{rpg-commentbox}

\newsect



\medskip
\begin{rpg-commentbox}{Setting game difficulty}
\begin{itemize}
    \item \textbf{Normal}: as described in the booklet;
    \item \textbf{Hard}: firefighters are \textbf{sleep deprived}; 
    
    The last 24h shift was a living hell and firefighters really don't know how people stay alive in this shitty station---built with second-hand and cheap equipment. 
    
    This is similar to the \textit{exhaustion} game mechanics,
    but \texttt{\textbf{STAMINA}} rolls are called at the DM discretion such as when you are hiding from an alien and trying to stay awake at the same time. 
    
    Good luck with that ;)
\end{itemize}
\end{rpg-commentbox}




\begin{figure}
    \centering
    \includegraphics[width=.55\textwidth]{img/bg/working-joe.png}
    \label{fig:refinery}
\end{figure}

\clearpage


\begin{figure*}
    \centering
    \includegraphics[width=1.0\textwidth]{img/bg/station.png}
\end{figure*}



% \makebox[0pt][l]{%
%   \raisebox{-\totalheight}[0pt][0pt]{%
%     \includegraphics[width=1.30\textwidth]{img/bg/station.png}}}%









% \begin{figure*}
%     \centering
%     \includegraphics[width=1.0\textwidth]{img/bg/motion-tracker.png}
% \end{figure*}

% \newpage

% \begin{figure*}
%     \centering
%     \includegraphics[width=1.0\textwidth]{img/bg/torch.png}
%     \label{fig:refinery}
% \end{figure*}

% \newpage

% \begin{figure*}
%     \centering
%     \includegraphics[width=1.0\textwidth]{img/bg/working-joe.png}
%     \label{fig:refinery}
% \end{figure*}

\chapter{Characters}


\section{Prisoners}


Why the prisoners don't just blow up the place? Well, truth to be told most of them are indeed terrorists and they have plans to either escape, gain control of the station as a new rendezvous point, or take revenge on both United Americas or  Weyland-Yutani. Let's just all accept any plausible excuses and have some fun playing RPG. 



\begin{rpg-commentbox}{Andy Myers}
    ``Kid'': a drug dealer. You have dwarfism and, over the years, you learned how to use this so that you passed unnoticed through marshals and law enforcement. 
\end{rpg-commentbox}

\begin{rpg-commentbox}{Fredy Cooperr}
    Officer: one of the most fearsome terrorists on anchor point station 2. You fought against corporate greed and blew several Weyland-Yutani installations before arrest. 
\end{rpg-commentbox}


\begin{rpg-commentbox}{Henrique Santiago}
    Smuggler: you trafficked people for organs extraction. It suffices to say that your lack any moral restraints.
\end{rpg-commentbox}

\begin{rpg-commentbox}{Hite Chyio}
    company agent: you were a Yakuza boss. You have blackmail information on several politicians and Weyland-Yutani. Kathia Rison is on her way to get you out of this hell. 
    
\end{rpg-commentbox}

\begin{rpg-commentbox}{Ivan Beschastnikh}
    marine: "dishonorably discharged" you do not talk about the reason why they threw you in this place either as a punishment worst than death or as a last favor. You are simply ruthless and think that in 3 to 6 months you will be back in the military under some black-ops team.
\end{rpg-commentbox}

\begin{rpg-commentbox}{Jone Heson}
    roughneck: you were wrongly accused for a crime you did not commit. You are a hardworking person and you had hopes that Weyland-Yutani lawyers would reach out and save you. Until then, you have to blend in and survive.
\end{rpg-commentbox}

\begin{rpg-commentbox}{Nolan Alder}
    scientist: a bio-terrorist. Before being caught, you left a sample of Chemical Agent A0-3959X.91–15 in a hidden stash. You plan to escape and launch a bio attack against anchor point station 2
\end{rpg-commentbox}

\newsect

\section{Lasalle Bionational crew}

Disguised as Weyland-Yutani annual inspection crew, the Lasalle plans to rescue one of the prisoners who has valuable information against their competitors.

\begin{rpg-commentbox}{Aleki Bowma}
    marine: chief of security of the Lasalle Bionational crew
    
\end{rpg-commentbox}


\begin{rpg-commentbox}{Barby Lopez}
    pilot: you brought Kathia on her corporate mission though you know the fame of Pallas and had you had the opportunity, you wouldn't think twice about shutting down the place
    
\end{rpg-commentbox}


\begin{rpg-commentbox}{Kathia Rison}
    , company agent: on your way to extract Hite from Pallas and get the hell out of this god forsaken place
    
\end{rpg-commentbox}


\begin{rpg-commentbox}{Tery Marte}
    roughneck reporter: an activist, you have a hidden camera in your right arm and you falsified paperwork to present yourself as a psychologist. You plan to evaluate the prisoners as an excuse to get first hand data on Pallas and use that for blackmailing or to shut the place down
\end{rpg-commentbox}

\newsect


\section{Pallas crew}

\begin{rpg-commentbox}{Joyce Ardson}
    roughneck: the chief engineer of the station
\end{rpg-commentbox}

\begin{rpg-commentbox}{Victor Macbeth}
    medic: chief medic on the station
\end{rpg-commentbox}

\begin{rpg-commentbox}{Mik Elson}
    marine: chief of security 
\end{rpg-commentbox}

\begin{rpg-commentbox}{Abdul Mageed}
    colonial marshal: a pragmatic. You run the entire station making sure that prisoners don't kill each other (more than the necessary to blow steam off). You turn a blind-eye to people wrongly accused thinking that the few false positives justify all the greater good.
\end{rpg-commentbox}

\newsect

\section{NPCs}

In case you need coadjuvants.

\begin{rpg-commentbox}{random prisoners}
    \begin{itemize}
        \item Mito Yoran
        \item Edzuk Shiro
        \item Bomba Andres
        \item Edison Harcrow
        \item Brom Tower
        \item Delphine Ryant
    \end{itemize}

\end{rpg-commentbox}


\begin{rpg-commentbox}{Random staff}
    \begin{itemize}
        \item Aleve Miroste
        \item Psycho Dante
        \item West Nylund
        \item Jocasta Belmont
        \item Geneva Macbeth
        \item Monte Jann
        \item Hawthorne Thoran
        \item Harrison Woldt
    \end{itemize}

\end{rpg-commentbox}

\newsect
\chapter{Act I}



\begin{rpg-commentbox}{}

    Corporate people arrive at the station and \pc{Kathia} wants to bring a \pc{PC} back to Lasalle Bionational territory. She speaks with the station marshal Abdul Mageed and is allowed to take the elevator down to refinery levels where she can talk with the prisoner.

    Either all or part of her crew go down with her to ensure her security (Mik and Victor) or to play some role in the extraction (Tery and Aleki). 
    
\end{rpg-commentbox}


% \medskip
% \begin{rpg-commentbox}{Lasalle Objectives}
% If you plan to run the adventure using the crew make sure that either no one follows Kathia to prison OR that players are equally divided between station and prison levels.

% \begin{itemize}
%     \item Potential introductions

%     \item Negotiation between the Lasalle and Kohru crews. \textbf{manipulation} \textbf{command}
% \end{itemize}
% \end{rpg-commentbox}



\medskip
\begin{rpg-commentbox}{Prisoner Objectives}
For the prisoners, keep them busy dividing them either at the mining area or the refinery area. Some events to serve as ice breakers:

\begin{itemize}
    \item There is a solar glass panel that is not protecting against ultraviolet radiation in the refinery. Part of the crew should fix that. An entire shift can fix a small portion of the panel but not everything. This will have a role later on. \textbf{heavy machinery}

    \item Rumors spread that there is a synthetic amongst the prisoners and people plot how to rule out who is the synthetic to feed misinformation to it \textbf{observation}

    \item A potential murder attempt among the prisoners \textbf{close combat}

    \item Bartering among prisoners \textbf{manipulation}
\end{itemize}
\end{rpg-commentbox}


\newsect

\section{Station Lockdown}



\begin{rpg-commentbox}{}
    
    Within some of the lithium ore, there are parasite spores created by the engineers. When the ore is heat and pulverized in the refinery section, a small parcel of the parasites spread through the station ducts. Two will make their way to hosts in the refinery section while a third one will infect someone at the crew quarters. 
    
    The Kohru computer system, \textbf{Dexter}, warns about environmental hazard contamination and puts the station in lockdown. Increase stress level by 1
\end{rpg-commentbox}    


\medskip
\begin{rpg-commentbox}{Notes for potential escape plans}
\begin{itemize}
    \item The only person with a corporate key card is at the prison (Kathia). The card is necessary to turn the engines of the Lasalle Bionational ship back online if anyone plans to escape. The keycard is \textbf{either} in Kathia's possession or locked under some biometric safe with instructions to corrupt the card if anyone tries to crack it open. 

    \item It's also necessary to open the landing hatch. Potential solutions involve using the sulfuric acid in the refinery or blowing the thing open. Either option will take the air out of the refinery and if done before the keycard is retrieved, that will put all future events on a timer. I advise to run for air supply periodically.
\end{itemize}
\end{rpg-commentbox}


% \begin{quotation}
% \begin{small}
% \textit{}

% \textit{}    
% \end{small}    
% \end{quotation}
    


\newsect

\newpage

\section{Escape plan}


\begin{rpg-commentbox}{}
    Lock-down won't let the elevator between the levels work until environment measurements are back to standard levels.
    
    
    The prisoners can check air filters at several locations to make sure the station's system is not just malfunctioning. In fact this is not the first time that something similar has happened. At some point, one of the prisioners will be brought back to medical where he start having convulsions...

\end{rpg-commentbox}

\begin{rpg-commentbox}{The Alien}
    This is also a good place for foreshadowing. Describe via radio communication some prisoners saying that some strange ore was found in the pulverizer or things that allude to an extra presence in the prison ward.
 \end{rpg-commentbox}




\begin{rpg-commentbox}{Infection stages}
    \begin{enumerate}
        \item convulsions and med bay
        \item splitting blood and more seizures
        \item host's death
    \end{enumerate}

    At this point, players are probably expecting the creature to burst out of the host. However, it won't happen and the spore will keep growing and merging its DNA with the hosts. Let them discuss and imagine whether this was a heart attack or something else. All initial attempts will point to a normal disease and equipment for an autopsy is only available at the upper level.  \textbf{medical aid}

\end{rpg-commentbox}





\newsect

\section{Heated discussion}


\begin{rpg-commentbox}{}
    
    Draw the players' attention somewhere else other than the dead body. If one player decides to stay behind and watch for the corpse, this may have several consequences in the game and a potential early death.
    
    -- A potential event is a discussion in the intercon between Kathia and Mageed, where Kathia desperately tries to convince him of escorting her out of the prison ward \textbf{manipulation}
    
    -- Tery can also reveal his hidden camera and say that he has footage that could ruin the Marshal's career

\end{rpg-commentbox}

\newsect



\begin{rpg-commentbox}{Fresh meat in the prison block}
    
    Negotiation to escape the prison ward is not successful. When players get back to the dead body's location, it is missing. Increase stress level by 1.
    
    There is a trail of blood leading into either the refinery direction or the mines. If one of the players stayed behind they witness:
    
    \textit{
    ``The alien is parasitic. It is still feeding from the host and using its motor system while it grows stronger. At this stage, players can see a protuberance in the person's back, exposed insect like tendrils ripping through the flesh, broken bones, and four long and thin spider-like appendages sprouting from the the host's back. The creature uses these sharp appendages to attack and kill potential threats or to infect new victims''.
    } 

    \medskip
    
    
    Roll a combat scene between any players that stayed behind and the creature. \textbf{Alien is at stage I (host attached)}
    
    If attacked and threatened, the creature can detach from the host and escape thorough an air duct. It will eventually find a new host in the mines.
    
    

\end{rpg-commentbox}    


\begin{rpg-commentbox}{Corporate security comes first}
    Kathia will persuade the prisoners to safeguard her and take her out of the station. Any prisoners protecting her will receive new identities in Lasalle Bionational controlled space.
\end{rpg-commentbox}
    


\begin{rpg-commentbox}{End of Act}
    \textbf{Act 1 ends when prisoners notice the missing body OR fight the stage I alien.}
 \end{rpg-commentbox}
\chapter{Act II}




\section{Where the hell is the body?}


\begin{rpg-commentbox}{The Alien}
   The creature controls the host to seek a dark and quiet place where it can grow stronger.

    On its path, it will infect two more prisoners (NPCs) making a larva exo-parasite grow on the host's stomach. 
    The larvas are in stealth mode and they need to infect a new host to reach stage II, i.e., exo-host.
\end{rpg-commentbox}


\newsect

\section{Escape plan}


\begin{rpg-commentbox}{Protect Kathia}
    Ideally, Kathia has left her corporate card in a biometric safe, so her protection is paramount to any escape plan.

    Mageed contacts prisoners via intercom and says that the crew is working on a escape plan. They will use Kathia's ship and for that, they need to bypass the lockdown.
    Prisoners are asked to gather enough mining explosives and bring them to the reactor core at the refinery. A handful of explosives can damage the reactor and cut power in the station, that will deactivate the lockdown. However, prisoners will need EVA suits as the air in the station will soon deplenish.
 \end{rpg-commentbox}


 \begin{rpg-commentbox}{Grab Explosives at the mine}
    
 \end{rpg-commentbox}


 
 \begin{rpg-commentbox}{Bring explosives to reactor}
    At this point in time, if one of the larvas is still alive and the PCs did not encounter it, people will hear agonizing screams as the larva infects a new host. 

    Something cool for this one is to have the host have a welding torch. The lava won't use it but it hangs in a loosely held bandolier making a lot of noise while the exo-parasite walks the refinery. This can make for some interesting scenes as for example, players setting the explosives in the reactor core while listening to the crackling sound of metal against the crate floor.
 \end{rpg-commentbox}

 \begin{rpg-commentbox}{Load cargo elevator with sulfuric acid}
    Needed to crack safe open without corrupting the corporate card

    Needed to destroy hangar bay door.
 \end{rpg-commentbox}

 \newsect

 \section{What is happening upstairs?}
 
 
 \begin{rpg-commentbox}{}
     From time to time Mageed gets in the intercom and give instructions.

     At some point, let him say hang-on, we got a situation here.

     All communications after this only return hiss static.
  \end{rpg-commentbox}
\chapter{Act III}





\begin{rpg-commentbox}{}
    
    Things upstair might have gone south. Even though the crew has side-arms, and a weapons-locker, the creature over took them by surprise. In fact, the alien at this act is at its full potential, stage III. 

\end{rpg-commentbox}    



\section{War zone}



\begin{rpg-commentbox}{}
    
    \textit{
    ``Red lights flash through the hangar. There is blood everywhere and disembodied members. Green goo merges itself with recognizable human blood and the silence is only broken by heavy footsteps of something too large walking in the hangar bay...''.
    } 

    \medskip
\end{rpg-commentbox}    



\begin{rpg-commentbox}{Let's get the hell out of here}
   The PCs must fetch the corporate card at the safe on the crew quarters. \textbf{mobility} vs \textbf{observation}

   When they reach the safe, they discover that it has been cracked open and the card is not there.

   \medskip

    The players might also want to grab side arms or heavy weapons at the weapons locker. For the heavy weapons, they need Mageed's card.
    Mageed is not found anywhere.
\end{rpg-commentbox}



\begin{rpg-commentbox}{What happened here?}
    If players decide to view the logs in the mainframe, they discover that part of the crew has locked themselves in the space shuttle and are waiting 
    for the prisoners to blow the hatch so they can escape instead. The prisoners can radio communicate with any survivors in the ship \textbf{manipulation} \textbf{command}

    For this to work, there has to be something in favor of the players, e.g., all pilots are dead and the only person with piloting skills is among the players; in desperation to escape the crew forgot to open some magnetic grips attached to the ship due to security protocols of the hangar. \textbf{heavy machinery}

    Make all negotiations stressful while the alien havocs/or camly walks through the hangar. If players take too long, the Praetorian will produce exo-parasites (Stage I alien) and drop them on the station so they can hunt new hosts.
 \end{rpg-commentbox}

 
\begin{rpg-commentbox}{Do we really need to kill this thing?}
    Players may get clever and crawl through the ducts and put an explosive on the hangar door sucking the the alien out of the station; use some of the heavy weapons and diversions to not confront the creature directly, etc.

    Reward clever thinking but punish time lost if decisions are not made fast enough. For example, if there is too much discussion, have a stage I alien jump from a duct and try to attach itself to one of the talkative players. 
 \end{rpg-commentbox}

\newsect 


\begin{rpg-commentbox}{Final showdown}
    Run how the players open the hangar door and negotiate with the Kohru crew. Combat may take a while, and this is intended to be a short one-shot, so tweek the alien stats if need be. Remember that act 3 has the resolution of many personal and conflicting agendas!
 \end{rpg-commentbox}


\begin{rpg-commentbox}{End of Act}
    \textbf{Act 3 ends with the players' escape or death.}
 \end{rpg-commentbox}

\section{Epilogue} 
 
\begin{rpg-commentbox}{}


\begin{itemize}
    \item Will Lasalle send a crew to gather alien DNA before word spreads out?
    \item Will the reporter release footage of his recordings or blackmail someone?
    \item Do the prisoners take control of the ship or are they out-matched by the surviving crew?
\end{itemize}    

    \medskip

    \textbf{Narrate the epilogue and thank everyone for their time.}
    \end{rpg-commentbox}

    \newsect

    
\chapter{Epilogue}


\begin{rpg-commentbox}{}


\begin{itemize}
    \item Will Lasalle send a crew to gather alien DNA before word spreads out?
    \item Will the reporter release footage of his recordings or blackmail someone?
    \item Do the prisoners take control of the ship or are they out-matched by the surviving crew?
\end{itemize}    

    \medskip

    \textbf{Narrate the epilogue and thank everyone for their time.}
 \end{rpg-commentbox}

 \newsect

 \section{Personal Agendas}


 
\begin{rpg-commentbox}{Andy Myers the drug dealer}
    \begin{enumerate}[label=Act \arabic*]
        \item Sell drugs from his personal stash to two inmates
        \item Find and convince someone to protect him
        \item Grab medical drugs to make his stress decrease
    \end{enumerate}
\end{rpg-commentbox}

\begin{rpg-commentbox}{Fredy Cooperr the terrorist}
    \begin{enumerate}[label=Act \arabic*]
        \item Make sure the other prisoners acknowledge his command
        \item Grab one land mine for his own personal usage
        \item Kill the marshal
    \end{enumerate}
\end{rpg-commentbox}


\begin{rpg-commentbox}{Henrique Santiago the smuggler}
    \begin{enumerate}[label=Act \arabic*]
        \item Grab smuggling supplies from one of the guard officers
        \item Make some improvised weapon powerful enough to kill the alien
        \item Get to the ship
    \end{enumerate}
\end{rpg-commentbox}

\begin{rpg-commentbox}{Hite Chyio the Yakuza boss}
    \begin{enumerate}[label=Act \arabic*]
        \item Discover that Kathia is there to rescue him
        \item Discover hidden camera on Tery
        \item Grab data from hidden camera for personal usage
    \end{enumerate}
    
\end{rpg-commentbox}

\begin{rpg-commentbox}{Ivan Beschastnikh the ex-soldier}
    \begin{enumerate}[label=Act \arabic*]
        \item Get enough nails to load his improvised shotgun
        \item Make sure Kathia survives
        \item Grab a fucking heavy-weapon
    \end{enumerate}
\end{rpg-commentbox}

\begin{rpg-commentbox}{Jone Heson the roughneck}
    \begin{enumerate}[label=Act \arabic*]
        \item Protect X from other prisoners
        \item Make some improvised weapon powerful enough to kill the alien
        \item Get to the ship
    \end{enumerate}
\end{rpg-commentbox}

\begin{rpg-commentbox}{Nolan Alder the  bio terrorist}
    \begin{enumerate}[label=Act \arabic*]
        \item Steal some of the medical supplies from the crew medic
        \item Discover the alien weakness
        \item Grab some of the spores from the Praetorian
    \end{enumerate}
\end{rpg-commentbox}

\newsect

\begin{rpg-commentbox}{Kathia Rison the company agent}
    \begin{enumerate}[label=Act \arabic*]
        \item Make sure Mageed and the others believe your crew works for Lasalle
        \item Survive
        \item Download data from mainframe
    \end{enumerate}
    
\end{rpg-commentbox}


\begin{rpg-commentbox}{Tery Marte the reporter}
    \begin{enumerate}[label=Act \arabic*]
        \item Record prisoner struggles
        \item Record alien \textbf{requires two slow actions}
        \item Make sure records are released somehow
    \end{enumerate}
\end{rpg-commentbox}




% \section{Kohru crew}

% \begin{rpg-commentbox}{Joyce Ardson}
%     roughneck: the chief engineer of the station
% \end{rpg-commentbox}

% \begin{rpg-commentbox}{Victor Macbeth}
%     medic: chief medic on the station
% \end{rpg-commentbox}

% \begin{rpg-commentbox}{Mik Elson}
%     marine: chief of security 
% \end{rpg-commentbox}

% \begin{rpg-commentbox}{Abdul Mageed}
%     colonial marshal: a pragmatic. You run the entire station making sure that prisoners don't kill each other (more than the necessary to blow steam off). You turn a blind-eye to people wrongly accused thinking that the few false positives justify all the greater good.
% \end{rpg-commentbox}
% \chapter{Appendix}



\begin{figure}
    \centering
    \includegraphics[width=.45\textwidth]{img/stage-I-bg.png}
    \label{fig:stage-1}
    \caption*{Stage I - Exo-parasite}
\end{figure}

\clearpage

\begin{figure}
    \centering
    \includegraphics[width=.45\textwidth]{img/stage-II-bg.png}
    \label{fig:stage-2}
    \caption*{Stage II - Exo-host}
\end{figure}


\clearpage

\begin{figure}
    \centering
    \includegraphics[width=\textwidth]{img/stage-III-bg.png}
    \label{fig:stage-3}
    \caption*{Stage III - Exo-alien}
\end{figure}

% End document
\end{document}
